% SPDX-License-Identifier: CC-BY-NC-SA-3.0

% Copyright 2022-2023 KUNBUS GmbH

\documentclass[aspectratio=169]{beamer}

\usepackage{xfrac}
\usepackage{minted}
\usepackage{xcolor}
\definecolor{LightGray}{gray}{0.9}

\usetheme {default}
\setbeamertemplate{navigation symbols}{}

\AtBeginSection[]{
  \begin{frame}
  \vfill
  \centering
  \begin{beamercolorbox}[sep=8pt,center,shadow=true,rounded=true]{title}
    \usebeamerfont{title}\insertsectionhead\par%
  \end{beamercolorbox}
  \vfill
  \end{frame}
}

\graphicspath{{images/}{\main/images/}}

% Title page details
\title {Introduction to device trees}
\subtitle{and device tree overlays}
%\author{latex-beamer.com}
%\institute{KUNBUS GmbH}
%\date{\today}

\newcommand{\sectiontitle}{}
\newcommand{\newsection}[1]{\renewcommand{\sectiontitle}{#1}\section{#1}}

% Image Logo
\logo{\includegraphics[width=2.5cm]{kunbus-logo.png}} 

\begin{document}

\begin{frame}
% Print the title page as the first slide
\titlepage
\end{frame}

% Presentation outline
\begin{frame}{Outline}
    \tableofcontents
\end{frame}

\newsection{Why do we need device trees?}
% Lists in beamer (Itemize)
\begin{frame}{Discoverable Hardware}
\begin{itemize}
    \item Devices we can discover and identify
    \item Well-known examples:
    \begin{itemize}
    	\item USB devices
    	\item PCI(e) devices
    \end{itemize}
\end{itemize}
\end{frame}

\begin{frame}{Discoverable Hardware}
\begin{block}{USB}
\begin{enumerate}
	\item Scan bus
	\item Enumerate devices
	\item Identify devices
	\item Bind driver to devices
\end{enumerate}
\end{block}
\end{frame}

\begin{frame}{Non-Discoverable Hardware}
\begin{itemize}
	\item Devices which have no way to communicate\\
	(LEDs, Buttons, \ldots)
	\item Devices which can't identify them self's\\
	(many I$^2$C and SPI devices)
	\item RAM, SRAM, ROM
    \item Other memory mapped devices (most controllers in ARM SoCs)
\end{itemize}
\end{frame}

\begin{frame}{Non-Discoverable Hardware}{Memory Mapped I/O}
\begin{figure}
    \centering
    \begin{overprint}
    \includegraphics<1>[width=\textwidth,height=0.6\textheight,keepaspectratio]{mmio_01}
    \includegraphics<2>[width=\textwidth,height=0.6\textheight,keepaspectratio]{mmio_02}
    \includegraphics<3>[width=\textwidth,height=0.6\textheight,keepaspectratio]{mmio_03}
    \end{overprint}
\end{figure}
\end{frame}

\newsection{What is a device tree?}
\begin{frame}{What is a device tree \ldots}
A device tree is an abstract hardware description.
\begin{itemize}
    \item This description should contain all relevant information about the hardware.
    \item Every device has its own device tree entry (node)
\end{itemize}
\end{frame}

\begin{frame}{\ldots and how is it used by Linux?}
\begin{enumerate}
    \item The bootloader loads a so called flattened device tree (fdt) blob
    \item It copies the fdt into the memory
    \item It might manipulate devictree entries (typically memory size is added)
    \item The bootloader starts the kernel and gives it a pointer to the device tree
    \item The kernel walks through the device tree and loads the matching drivers for the device tree entries
    \item The drivers read additional configuration parameters from the device tree entries
\end{enumerate}
\end{frame}

\newsection{What does a device tree look like?}
\begin{frame}[fragile]{A simplified device tree}{\sectiontitle}
\begin{columns}[T]
\column{0.5\textwidth}
\begin{minted}[fontsize=\tiny]{dts}
/ {
	compatible = "brcm,bcm2835";

	cpus {
		cpu@0 {
			device_type = "cpu";
			compatible = "arm,arm1176jzf-s";
		};
	};

	soc {
		...
	};

	arm-pmu {
		compatible = "arm,arm1176-pmu";
	};
};
\end{minted}
\column{0.5\textwidth}
\begin{block}{node}
A node is defined inside curly braces.
\begin{minted}[fontsize=\small, firstline=2]{dts}
/ {
{...}
\end{minted}
\end{block}
\begin{block}{node identifier}
Every node has a unique identifier.
\begin{minted}[fontsize=\small, firstline=2]{dts}
/ {
cpus {...};
\end{minted}
\end{block}
\begin{block}{root node}
The root node is always named \verb|/|.
\begin{minted}[fontsize=\small]{dts}
/ {...};
\end{minted}
\end{block}
\end{columns}
\end{frame}

\begin{frame}[fragile]{A simplified device tree: Part Deux}{\sectiontitle}
\begin{columns}[T]
\column{0.5\textwidth}
\begin{minted}[fontsize=\tiny]{dts}
soc {
	uart0: serial@7e201000 {
		compatible = "arm,pl011", "arm,primecell";
		reg = <0x7e201000 0x200>;
		...
	};
	sdhost: mmc@7e202000 {
		compatible = "brcm,bcm2835-sdhost";
		reg = <0x7e202000 0x100>;
		...
	};
	spi: spi@7e204000 {
		compatible = "brcm,bcm2835-spi";
		reg = <0x7e204000 0x200>;
		...
	};
	i2c0: i2c@7e205000 {
		compatible = "brcm,bcm2835-i2c";
		reg = <0x7e205000 0x200>;
		...
	};
	dpi: dpi@7e208000 {
		compatible = "brcm,bcm2835-dpi";
		reg = <0x7e208000 0x8c>;
		...
	};
};
\end{minted}
\column{0.5\textwidth}
\begin{block}{node identifier convention}
If it makes any sense the identifier of a node should have its \emph{address}
encoded. This is done with an \verb|@|.
\begin{minted}[fontsize=\small, firstline=2]{dts}
/ {
serial@7e201000 {...};
\end{minted}
\end{block}
\begin{block}{node label}
The optional node label is placed before the node identifier. And is used to reference later e.g. from other nodes.
\begin{minted}[fontsize=\small, firstline=2]{dts}
/ {
uart0: ...
\end{minted}
\end{block}
\end{columns}
\end{frame}

\begin{frame}[fragile]{A simplified device tree: 33\textsuperscript{\sfrac{1}{3}}}{\sectiontitle}
\begin{columns}[T]
\column{0.5\textwidth}
\begin{minted}[fontsize=\tiny]{dts}
soc {
	uart0: serial@7e201000 {
		compatible = "arm,pl011", "arm,primecell";
		reg = <0x7e201000 0x200>;
		...
	};
	sdhost: mmc@7e202000 {
		compatible = "brcm,bcm2835-sdhost";
		reg = <0x7e202000 0x100>;
		...
	};
	spi: spi@7e204000 {
		compatible = "brcm,bcm2835-spi";
		reg = <0x7e204000 0x200>;
		...
	};
	i2c0: i2c@7e205000 {
		compatible = "brcm,bcm2835-i2c";
		reg = <0x7e205000 0x200>;
		...
	};
	dpi: dpi@7e208000 {
		compatible = "brcm,bcm2835-dpi";
		reg = <0x7e208000 0x8c>;
		...
	};
};
\end{minted}
\column{0.5\textwidth}
\begin{block}{compatible}
The \verb|compatible| property get a list of strings. Those strings are matched with the drivers. The first match wins.
\begin{minted}[fontsize=\small, firstline=2]{dts}
/ { node { // not shown on the slide
compatible = "arm,pl011", "arm,primecell"
\end{minted}
\end{block}
\begin{block}{reg}
The \verb|reg| property is (usually) used by the driver to identify the correct (base) address of the device.
\begin{minted}[fontsize=\small, firstline=2]{dts}
/ { node { //not shown on the slide
reg = <0x7e201000 0x200>;
\end{minted}
\end{block}
\end{columns}
\end{frame}

\begin{frame}[fragile]{A simplified device tree: Episode IV}{\sectiontitle}
\begin{columns}[T]
\column{0.5\textwidth}
\begin{minted}[fontsize=\tiny]{dts}
soc {
	uart0: serial@7e201000 {
		compatible = "arm,pl011", "arm,primecell";
		...
		status = "okay"
	};
	sdhost: mmc@7e202000 {
		compatible = "brcm,bcm2835-sdhost";
		...
		status = "okay"
	};
	spi: spi@7e204000 {
		compatible = "brcm,bcm2835-spi";
		...
		status = "disabled"
	};
	i2c0: i2c@7e205000 {
		compatible = "brcm,bcm2835-i2c";
		...
		status = "disabled"
	};
	dpi: dpi@7e208000 {
		compatible = "brcm,bcm2835-dpi";
		...
		status = "disabled"
	};
};
\end{minted}
\column{0.5\textwidth}
\begin{block}{status}
The \verb|status| property is used to define if a node should be active or not.
If the property is set to \verb|"okay"| the node seen as active and a driver
can process the node. If the property is set to \verb|"disabled"| the node is
seen as inactive. So no driver will process the node.
\begin{minted}[fontsize=\small, firstline=2]{dts}
/ { node { // not shown on the slide
status = "disabled";
\end{minted}
\end{block}
\end{columns}
\end{frame}

\end{document}
