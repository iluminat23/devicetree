% SPDX-License-Identifier: CC-BY-NC-SA-3.0

% Copyright 2022 KUNBUS GmbH

\documentclass[aspectratio=169]{beamer}

\usepackage{minted}
\usepackage{xcolor}
\definecolor{LightGray}{gray}{0.9}

\usetheme {default}
\setbeamertemplate{navigation symbols}{}

\AtBeginSection[]{
  \begin{frame}
  \vfill
  \centering
  \begin{beamercolorbox}[sep=8pt,center,shadow=true,rounded=true]{title}
    \usebeamerfont{title}\insertsectionhead\par%
  \end{beamercolorbox}
  \vfill
  \end{frame}
}

\graphicspath{{images/}{\main/images/}}

% Title page details
\title {Introduction to device trees}
\subtitle{and device tree overlays}
%\author{latex-beamer.com}
%\institute{KUNBUS GmbH}
%\date{\today}

% Image Logo
\logo{\includegraphics[width=2.5cm]{kunbus-logo.png}} 

\begin{document}

\begin{frame}
% Print the title page as the first slide
\titlepage
\end{frame}

% Presentation outline
\begin{frame}{Outline}
    \tableofcontents
\end{frame}

\section{Why do we need device trees?}
% Lists in beamer (Itemize)
\begin{frame}{Discoverable Hardware}
\begin{itemize}
    \item Devices we can discover and identify
    \item Well-known examples:
    \begin{itemize}
    	\item USB devices
    	\item PCI(e) devices
    \end{itemize}
\end{itemize}
\end{frame}

\begin{frame}{Discoverable Hardware}
\begin{block}{USB}
\begin{enumerate}
	\item Scan bus
	\item Enumerate devices
	\item Identify devices
	\item Bind driver to devices
\end{enumerate}
\end{block}
\end{frame}

\begin{frame}{Non-Discoverable Hardware}
\begin{itemize}
	\item Devices which have no way to communicate\\
	(LEDs, Buttons, \ldots)
	\item Devices which can't identify them selfs\\
	(many I$^2$C and SPI devices)
	\item RAM, SRAM, ROM
    \item Other memory mapped devices (most controllers in ARM SoCs)
\end{itemize}
\end{frame}

\begin{frame}{Non-Discoverable Hardware}{Memory Mapped I/O}
\begin{figure}
    \centering
    \begin{overprint}
    \includegraphics<1>[width=\textwidth,height=0.6\textheight,keepaspectratio]{mmio_01}
    \includegraphics<2>[width=\textwidth,height=0.6\textheight,keepaspectratio]{mmio_02}
    \includegraphics<3>[width=\textwidth,height=0.6\textheight,keepaspectratio]{mmio_03}
    \end{overprint}
\end{figure}
\end{frame}

\section{What is a device tree?}
\begin{frame}{What is a device tree \ldots}
A device tree is an abstract hardware description.
\begin{itemize}
    \item This description should contain all relevent information about the hardware.
    \item Every device has its own device tree entry (node)
\end{itemize}

\end{frame}

\begin{frame}{\ldots and how is it used by Linux?}
\begin{enumerate}
    \item The bootloader loads a so called flattend device tree (fdt) blob
    \item It copies the fdt into the memory
    \item It might manipulate devictree entries (typlically memory size is added)
    \item The bootloader starts the kernel and gives it a pointer to the device tree
    \item The kernel walks through the device tree and loads the matching drivers for the device tree entries
    \item The drivers read additional configuration parameters from the device tree entries
\end{enumerate}
\end{frame}
\end{document}
